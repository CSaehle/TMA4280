\section{Discussion}

\subsection{Solving a different Poisson problem with another $f$}

We implemented two variations on the parallel program, using a different $f$ on
the right hand side of the Poisson equation. First, we changed the equation to
the smooth function, $f(x, y) = e^x \cdot sin ( 2 \pi x) \cdot sin (\pi y)$.
Secondly, we changed the equation to make f represent two point sources, $f = 0$
in the whole domain except for $f(0, 0) = 1$ and $f(m-1, m-1) = -1$. The only
modifications we needed to make to the program was the way $f$ was generated.

\subsection{Supporting Non-zero Boundary Conditions}

Supporting non-zero boundary conditions would require us to change Equation 7 to

\begin{equation}
  (\underline{TU} + \underline{UT})i,j = h^2f_{i,j} + \underline U = \underline{G}
\end{equation}



\subsection{Generalizing to Rectangular Domains}
Generalizing this implementation to rectangular domains has no effect on the
method. The only difference would thus be the grid. As such, the only difference
in the implementation is that the transposing has to be explicit grid
dimensions, and that one uses the rectangular width and rectangular height
properly at correct positions.
